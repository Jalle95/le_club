\section{Chancekort}
Nogle af felterne på \LeClub\ er røde. Ender en karakter på et rødt felt må karakterens hold trække et chancekort. Chancekortet læses højt og afvikles før turen gives videre. Alle kort \textbf{skal} afvikles! Kan et hold ikke afvikle kortet udgår holdet af spillet. Efter et kort er afviklet, skal det enten gemmes (fremgår af kortet) eller lægges i bunken med brugte chancekort. Når der ikke er flere chancekort genbruges bunken med brugte kort efter denne er blevet blandet.
\subsection{Telefonnumre}
Trækker en karakter et chancekort med et telefonnummer gemmes kortet indtil opkaldet er udført. Alle karakterer på holdet kan udføre opkaldet, men man kan kun lave opkald på toilettet. Et opkald resulterer altid i venskab mellem karakteren der udfører opkaldet og personen der ringes til. Der ringes før et terningekast, men opkaldet tæller ikke for en tur.
\begin{itemize}
\item[\faPhone] 66 66 66 66 : Dørmanden
\item[\faPhone] 11 22 33 44 : Partydværgen
\item[\faPhone] 12 13 14 15 : Stripperen
\end{itemize}
\subsection{Aggressivitet}
Nogle chancekort gør karakteren der trækker kortet aggressiv. Aggressivitet er et tveægget sværd, der kan være nøglen til at vende eller lukke spillet. For aggressive karakterer gælder der følgende: En aggressiv karakter kan smide en anden karakter ud ved blot at lande på det felt hvor karakteren står. En aggressiv karakter der ses af dørmanden smides straks ud af klubben. Trækker en aggressiv karakter endnu et kort der giver aggressivitet har kortet ingen effekt og det lægges blot i bunken med brugte chancekort. En karakter mister sin aggressivitet hvis karakteren bliver smidt ud eller får et glas vand (se baren).

%\ThisTileWallPaper{\paperwidth}{\paperheight}{rulebook_page.png}

\subsection{Studiekortet}
I blandt chancekortene gemmer det legendariske studiekort sig. Studiekortet giver en karakter dobbelt op på alt i baren. Det er ikke muligt at \textit{give} andre karakterer på sit hold studiekortet. Det er til gengæld muligt at stjæle studiekortet fra karakteren der har det. Dette gøres ved at flytte en karakter over og forbi karakteren med studiekortet. Det er muligt at stjæle studiekortet overalt på \LeClub . Står karakteren med studiekortet på det sidste felt i bagindgangen eller på toilettet er det nok at rykke op på siden af karakteren. Karakteren der får kortet stjålet i dette tilfælde skal først rykke fra feltet for at genstjæle kortet. Når karakteren med studiekortet går op i VIP-loungen udgår studiekortet af spillet. Det lægges \textbf{ikke} i bunken med brugte chancekort.
\subsection{The Golden Ticket}
The Golden Ticket er en alternativ måde at få sine karakterer op i VIP-loungen. Kortet har kun en effekt hvis karakteren der trækker kortet har optjent dobbelt så meget fame som det kræves at komme i VIP-loungen. Har karakteren ikke optjent nok fame lægges kortet blot i bunken med brugte chancekort. Har karakteren optjent nok fame, gemmes kortet ved holdet, indtil det er brugt eller at karakteren smides ud. I dette tilfælde lægges kortet i bunken med brugte chancekort. 

Effekten af The Golden Ticket er spilvindende. Når en karakter med billetten til VIP-loungen med tilstrækkelig fame (\textbf{ikke} dobbelt fame), kan vedkommende tage alle holdets karakterer der er på klubben med op i VIP-loungen, ligegyldigt hvor karaktererne er og hvor meget fame de har optjent. 
\raggedcolumns

