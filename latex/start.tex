\section{Spillets start}
Hvert hold placerer én af deres karakterer på feltet foran \LeClub . Alle andre karakterer starter uden for klubben på det bagerste felt. Alle knapper en frisk breezer op. Den yngste spiller starter. 

For at rykke en karakter, der står uden for klubben skal der slås 5 eller 6 med terningen. Uden for klubben rykkes der ét felt ad gangen. For at rykke en karakter ind fra det sidste felt af fortovet til det første felt på klubben, markeret med en blå stjerne, skal der betales entre. Karakteren, der er betalt entre for, får et stempel og rykkes ind på klubbens første felt efter der er betalt. Stemplet giver adgang til klubben resten af spillet. Entre betales før terningekast. Har et hold ikke en karakter på klubben får man tre forsøg til at slå enten 5 eller 6 med terningen.

En karakter på klubben rykkes det antal felter øjnene på terningen viser. Karakteren kan flyttes både frem og tilbage, men ikke begge dele på samme tur. Ved forgrening rykkes karakteren valgfrit ad en af vejene. Har et hold mere end én karakter på klubben kan holdet selv vælge hvilken karakter der rykkes. Kun én karakter kan rykkes per tur. Når en karakter er sluppet på et gyldigt felt er holdets tur slut. Turen går efterfølgende i negativ omløbsretning. 

Det er i udgangspunktet ikke muligt for to karakterer at stå på samme felt. Spærrer en karakter for at lande på et felt, rykkes karateren i bevægelse til det efterfølgende felt efter normale regler for at rykke. På toilettet og i bagindgangen kan to karakterer stå på samme felt. Dette har ingen konsekvens. Enderne i bagindgangen og på toilettet virker som fuldt stop og kan stoppe alle karakterer lige gyldigt hvor langt karakteren ville rykke videre.